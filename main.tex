% !Mode:: "TeX:UTF-8"
% \documentclass[master,color,AutoFakeBold=true]{buaathesisproposal}
\documentclass[master,AutoFakeBold=true]{buaathesisproposal}

% 参考文献
% 参考文献输出方式,numerical为按照出现顺序,authoryear为按照作者姓名和年份
\bibliographystyle{gbt7714-numerical}

\begin{document}

% 用户信息
% !Mode:: "TeX:UTF-8"

% 学院中文名,中文不需要“学院”二字
\school
{(学院名)}

% 专业中文名
\major
{(专业名)}

% 论文中文标题
\thesistitle
% {北航开题报告\LaTeX{}模板}
{北京航空航天大学开题报告\LaTeX{}模板}

% 作者中文名
\thesisauthor
{(姓名)}

% 导师中文名及职称
\teacher
{(导师姓名)}
\teacherdegree
{(导师职称)}


% 中图分类号,可在 http://www.ztflh.com/ 查询
\category{(中图分类号)}

% 学号
\studentID{(xy1234567)}

% 报告日期
\proposaldate{(年)}{(月)}{(日)}



% 中英封面
\maketitle
% 目录
\makecontents

% 正文页码、页眉、页脚样式
\pagestyle{mainmatter}

% 正文
\section{选题依据}
主要阐述论文选题的背景与研究工作的意义。
\subsection{背景}

选题背景

\subsubsection{背景一}

研究生在导师的指导下,根据所研究的课题背景,进行开题报告选题。该选题应属于本学科范围。各单位导师应充分重视,指导、督促、检查学生做好开题报告及文献综述工作。同时要求研究生主动与导师沟通,积极、严谨地完成开题报告。

\subsubsection{背景二}

开题报告内容应包含:
(1)开题题目;
(2)选题依据,包括选题的意义、必要性和前沿性;
(3)文献综述,一般应针对研究方向或具体问题进行深入调研、对相关研究现状进行系统的分析、说明,并指出存在的问题或不足,阐述发展趋势并引出拟研究问题;
(4)研究方案,一般包括研究目标、研究内容、要解决的关键问题、拟采取的研究方法、技术路线等;
(5)预期目标和成果;
(6)学位论文实施计划;
(7)主要参考文献等。


硕士研究生的开题报告字数一般不少于1.5万字,博士研究生的开题报告一般不少于4万字。


\section{文献综述}

主要阐述研究方向国内外的研究现状、最新研究成果及不足之处。

\subsection{参考文献引用方法}

本项目使用 Bibtex 的方法管理参考文献。
BibTeX 是一种格式和一个程序,用于协调LaTeX的参考文献处理. BibTeX 使用数据库的的方式来管理参考文献. BibTeX 文件的后缀名为 .bib .

\subsubsection{形成 .bib 文件}

在 Google Scholar 等学术搜索引擎中搜索文献,点击引用,选择 BibTeX,复制文献信息,粘贴到 .bib 文件中即可。
例如,文献 \cite{kottwitz2011latex} 的 BibTeX 信息为
\begin{lstlisting}[
    caption={bib格式文件示例},
    label={code:bib-sample},
]
@book{kottwitz2011latex,
  title={LaTeX Beginner's Guide},
  author={Kottwitz, S.},
  isbn={9781847199867},
  url={http://books.google.com.hk/books?id=rB1Cb62dVnUC},
  year={2011},
  publisher={Packt Publishing}
}
\end{lstlisting}

\subsubsection{引用参考文献}

在正文中引用参考文献,使用 \verb|\cite{key}| 命令即可,其中 key 为 .bib 文件中文献的标识符。
例如,在正文中使用 \verb|\cite{kottwitz2011latex}| 命令,即可在正文中引用代码~\ref{code:bib-sample} 中的文献。

\subsubsection{编译参考文献}

编译参考文献,需要先编译 .tex 文件,再编译 .bib 文件,最后再编译 .tex 文件两次。
例如,使用 \verb|pdflatex| 编译 .tex 文件,使用 \verb|bibtex| 编译 .bib 文件,再次使用 \verb|pdflatex| 编译 .tex 文件,最后再次使用 \verb|pdflatex| 编译 .tex 文件。


\subsection{参考文献格式调整}

模板参考文献格式采用国家标准 GB/T 7714-2005 《信息与文献 参考文献著录规则》之中描述的格式。代码实现为 \href{http://github.com/CTeX-org/gbt7714-bibtex-style/}{CTeX-org/gbt7714-bibtex-style v2.1.5}。参考文献详细说明请见该项目 README.md。

参考文献提供两种排序方式,分别为 按出现顺序(默认)和 按作者姓名和年份,请按需在模板 main.tex 中的第7行做相应修改。

注意:根据 GB/T 7714-2005 中 8.4 节 出版项  中相关规定:

无出版地的中文文献著录 “出版地不详”,外文文献著录 “S.l.”
无出版者的中文文献著录 “出版者不详”,外文文献著录 “s.n.”
相应的解决方法为

若编译的参考文献条目中出现 “出版地不详” 或 “S.l.”,请在相应的 bib 条目中添加 address 相关信息
若编译的参考文献条目中出现 “出版者不详” 或 “s.n.”,请在相应的 bib 条目中添加 publisher 相关信息
实际使用中应避免出现 [S.l.]:[s.n.] 这样的著录形式。

\section{研究方案}
主要阐述研究目标、主要研究内容、拟采用的研究方法和可行性分析、可能的创新之处等。

\subsection{图片插入}

插入图片的方法如代码~\ref{code:fig-sample} 所示。
\begin{lstlisting}[
    caption={图片插入示例},
    label={code:fig-sample},
]
\begin{figure}[htbp]
    \centering
    \includegraphics[width=0.5\textwidth]{buaa_logo}
    \caption{BUAA LOGO}
    \label{fig:buaa_logo}
\end{figure}
\end{lstlisting}
其中,\verb|[htbp]| 为图片位置,\verb|h| 为 here,\verb|t| 为 top,\verb|b| 为 bottom,\verb|p| 为 page,\verb|\centering| 为居中,\verb|\includegraphics| 为插入图片,\verb|width=0.5\textwidth| 为图片宽度为正文宽度的一半,\verb|buaa_logo| 为图片文件名,\verb|caption| 为图片标题,\verb|label| 为图片标签,可用于引用图片。
上述代码运行结果如图~\ref{fig:buaa_logo} 所示。

\begin{figure}[htbp]
    \centering
    \includegraphics[width=0.5\textwidth]{buaa_logo}
    \caption{BUAA LOGO}
    \label{fig:buaa_logo}
\end{figure}


如果需要插入多张图片,可以使用 \verb|subfigure| 环境,如代码~\ref{code:subfig-sample} 所示。
\begin{lstlisting}[
    caption={子图插入示例},
    label={code:subfig-sample},
]
\begin{figure}[htbp]
    \centering
    \subfigure[BUAA LOGO1]{
        \includegraphics[width=0.3\textwidth]{buaa_logo}
        \label{fig:subfig-buaa_logo1}
    }
    \subfigure[]{
        \includegraphics[width=0.3\textwidth]{buaa_logo}
        \label{fig:subfig-buaa_logo2}
    }
    \caption{BUAA LOGO}
    \label{fig:subfig-sample}
\end{figure}
\end{lstlisting} 
其中,\verb|\subfigure| 为子图,\verb|[width=0.3\textwidth]| 为子图宽度为正文宽度的三分之一,\verb|buaa_logo| 为图片文件名,\verb|label| 为图片标签,可用于引用图片,\verb|caption| 为图片标题。
上述代码运行结果如图~\ref{fig:subfig-sample} 所示。

\begin{figure}[htbp]
    \centering
    \subfigure[BUAA LOGO1]{
        \includegraphics[width=0.3\textwidth]{buaa_logo}
        \label{fig:subfig-buaa_logo1}
    }
    \subfigure[]{
        \includegraphics[width=0.3\textwidth]{buaa_logo}
        \label{fig:subfig-buaa_logo2}
    }
    \caption{BUAA LOGO}
    \label{fig:subfig-sample}
\end{figure}

\subsection{表格插入}

插入表格的方法如代码~\ref{code:tab-sample} 所示。
\begin{lstlisting}[
    caption={表格插入示例},
    label={code:tab-sample},
]
\begin{table}[htbp]
    \centering
    \caption{表格示例}
    \label{tab:tab-sample}
    \begin{tabular}{ccc}
        \toprule
        1 & 2 & 3 \\
        \midrule
        4 & 5 & 6 \\
        7 & 8 & 9 \\
        \bottomrule
    \end{tabular}
\end{table}
\end{lstlisting}
其中,\verb|[htbp]| 为表格位置,\verb|h| 为 here,\verb|t| 为 top,\verb|b| 为 bottom,\verb|p| 为 page,\verb|\centering| 为居中,\verb|caption| 为表格标题,\verb|label| 为表格标签,可用于引用表格,\verb|tabular| 为表格,\verb|{ccc}| 为表格列数,\verb|\toprule| 为表格上边框,\verb|\midrule| 为表格中边框,\verb|\bottomrule| 为表格下边框。
上述代码运行结果如表~\ref{tab:tab-sample} 所示。

\begin{table}[htbp]
    \centering
    \caption{表格示例}
    \label{tab:tab-sample}
    \begin{tabular}{ccc}
        \toprule
        1 & 2 & 3 \\
        \midrule
        4 & 5 & 6 \\
        7 & 8 & 9 \\
        \bottomrule
    \end{tabular}
\end{table}

\subsection{公式插入}

插入公式的方法如代码~\ref{code:eq-sample} 所示。
\begin{lstlisting}[
    caption={公式插入示例},
    label={code:eq-sample},
]
\begin{equation}
    \label{eq:eq-sample}
    a^2 + b^2 = c^2
\end{equation}
\end{lstlisting}
其中,\verb|equation| 为公式,\verb|label| 为公式标签,可用于引用公式。
上述代码运行结果如公式~\ref{eq:eq-sample} 所示。

\begin{equation}
    \label{eq:eq-sample}
    a^2 + b^2 = c^2
\end{equation}


\section{预期目标和成果}
主要阐述论文的预期目标与学术论文、专利等成果。

\section{学位论文实施计划}
主要阐述论文详细进度安排。

\section{主要参考文献}
\nocite{*}
\bibliography{bibs}

\end{document}
